\chapter{Insiemi invarianti, Orbite e Omega-limiti}

\section{Orbite, punti fissi e periodici}
\begin{definition}[Orbita]
Definiamo l'\textbf{orbita positiva} di un'equazione di flusso $\phi$ come
\[\Oc^+(x)=\bigcup_{t\geq 0}\phi_t(x).\]
In modo simile definiamo l'\textbf{orbita negativa} e l'\textbf{orbita}.\\
La \textbf{traiettoria} di $x$ \`e il grafico $t\mapsto \phi_t(x)$ contenuto in $\R\times \R^d$\footnote{questa \`e la terminologia italiana. In inglese ''trajectory" corrisponde alla nostra orbita, non alla traiettoria.}.
\end{definition}

\begin{definition}[Punti fissi]
Un \textbf{punto fisso} $x\in\R^d$ \`e un punto tale che $F(x)=0$.
\end{definition}
\begin{remark}
L'orbita di un punto fisso \`e composta da un singoletto.
\end{remark}

\begin{definition}[Punti periodici]
Un punto $x\in\R^d$ \`e \textbf{periodico} se esiste $T>0$ tale che
\[\phi_{T+s}(x)=\phi_s(x)\quad \forall s\in\R\]
e $\phi_s(x)\neq x$ per ogni $s\in (0,T)$.\\
$T$ \`e detto \textbf{periodo minimo} di $x$.
\end{definition}
\begin{remark}
L'orbita di un punto periodico \`e data da
$\Oc(x)=\cpa{\phi_s(x)\mid s\in[0,T)}$.
\end{remark}

\begin{remark}[Simulazione]
Di solito per tracciare orbite usiamo il computer ma per natura di tali simulazioni, i dati iniziali (e l'integrazione stessa) non pu\`o essere assolutamente precisa. \`E quindi quasi impossibile disegnare algoritmicamente alcune orbite essenziali per comprendere il funzionamento del sistema (per esempio orbite periodiche, omo- ed eterocline e orbite limite tra due comportamenti). 
\end{remark}

\section{Insiemi invarianti}
\begin{definition}[Insiemi invarianti]
Un insieme $A\subseteq \R^d$ si dice \textbf{positivamente invariante} se per ogni $x\in A$, $\phi_t(x)\in A$ per ogni $t\geq 0$. In modo simile definiamo insiemi \textbf{negativamente invarianti} e \textbf{invarianti}.
\end{definition}

\begin{remark}[Le orbite sono invarianti]
Le orbite positive sono positivamente invarianti, le orbite negative sono negativamente invarianti e le orbite sono invarianti.
\end{remark}
\begin{proof}
Segue dal fatto che $\phi_{s+t}(x)=\phi_s(\phi_t(x))$.
\end{proof}

\begin{definition}[Integrale primo]
Una funzione $I:\R^d\to \R,\ I\in C^1(\R^d)$ \`e un integrale primo per $\dot x=F(x)$ se per ogni $x\in\R^d$
\[\dot I(x)\doteqdot \ps{\nabla I(x),F(x)}=\rbar{\dd{t}{}I(\phi_t(x))}_{t=0}=0.\]
\end{definition}



\begin{proposition}[Costruzione di insiemi invarianti da curve di livello]\label{CostruzioneInsiemiInvariantiCurveDiLivello}
Sia $c\in\R$ e $I:\R^d\to \R,\ I\in C^1$ tale che $\nabla I\res{\cpa{I=c}}\not\equiv 0$ e $\dot I\res{\cpa{I=c}}=0$, allora $\cpa{I=c}$ \`e invariante.
\end{proposition}
\begin{proof}
Sia $x_0\in I_c$. In particolare $\nabla I(x_0)\neq 0$. Senza perdita di generalit\`a supponiamo che l'ultima entrata sia non nulla, cio\`e $\pp{x_d}I(x_0)\neq 0$. Per il teorema della funzione implicita esistono un intorno $U$ di $x_0$ in $\R^d$ e una funzione $h:\R^{d-1}\to \R$ della stessa regolarit\`a di $I$ tali che\footnote{La mappa $\pi_{\neq x_d}$ \`e la proiezione ovvia da $\R^d$ a $\R^{d-1}$ che elimina l'ultima coordinata.}
\[I_c\cap U=\cpa{\ulx\in \pi_{\neq x_d}(U)\mid I(\ulx, h(\ulx))=c}.\]
A meno di restringere $U$, per il teorema sulla permanenza del segno supponiamo che $\pp{x_d}I\res{U}$ non si annulli mai.\\
Scriviamo $F=(\ulf,g)$ con $\ulf:\R^d\to \R^{d-1}$ e $g:\R^d\to\R$. Se $x_0=(\ulx_0,y_0)$, consideriamo il problema di Cauchy
\[\begin{cases}
\dot \ulx=\ulf(\ulx,h(\ulx))\\
\ulx(0)=\ulx_0
\end{cases}\]
e sia $\wt \psi_t(\ulx_0)$ la sua soluzione in un intorno di $\ulx_0$. Poniamo $\psi_t(x_0)=(\wt \psi_t(\ulx_0),h(\wt \psi_t(\ulx_0)))$ e verifichiamo che $\psi=\phi$. Chiaramente le condizioni iniziali coincidono, cos\`i come la derivata lungo le prime $d-1$ entrate. Per concludere basta dunque verificare che $\dd{t}h(\ulx)=g(\ulx,h(\ulx))$.\\
Derivando $I(\psi_t(x_0))=c$ troviamo 
\[\nabla I(\psi_t(x_0))\cdot \mat{\ulf(\ulx,h(\ulx))\\\dd{t}h(\ulx)}=0\]
e dalla condizione $\dot I(\psi_t(x_0))=0$ ricaviamo
\[\nabla I(\psi_t(x_0))\cdot \mat{\ulf(\ulx,h(\ulx))\\g(\ulx,h(\ulx))}=0.\]
Confrontando queste due equazioni ricaviamo
\[\pp{x_d}I(\psi_t(x_0))\pa{\dd{t}h(\ulx)-g(\ulx,h(\ulx))}=0,\]
ma poich\'e abbiamo supposto che $\pp{x_d}I\res{U}\neq 0$, ricaviamo l'uguaglianza cercata per un opportuno intorno di $x_0$.
\end{proof}

\begin{corollary}[Invarianza degli insiemi di livello di integrali primi]\label{InvarianzaInsiemiLivelloIntegraliPrimi}
Se $I$ \`e un integrale primo allora per ogni $c\in\R$ tale che $\nabla I\res{\cpa{I=c}}\neq0$ l'insieme
\[I_c=\cpa{I=c}\]
\`e invariante.
\end{corollary}


\begin{remark}[Intuizione geometrica]
Se $\nabla I(x)\neq 0$, la condizione $\dot I(x)=0$ afferma che $\nabla I(x)$ e $F(x)$ sono perpendicolari. Per il teorema delle funzioni implicite, se $x\in I_c$ abbiamo che un vettore normale a $I_c$ in $x$ \`e $\nabla I$, dunque se $F(x)$ vi \`e perpendicolare si ha che $F(x)\in T_x I_c$, in particolare l'orbita tende a procedere rimanendo nello spazio tangente, cio\`e resta contenuta in $I_c$.
\end{remark}

\section{Alpha e Omega limiti}
\begin{definition}[Alpha e Omega limiti]
Dato $x\in\R^d$ chiamiamo \textbf{$\omega$-limite} di $x$ l'insieme
\[\omega(x)=\cpa{y\in\R^d\mid \exists \{t_n\}\nearrow +\infty\ t.c.\ \lim_{k\to+\infty}\phi_{t_k}(x)=y}.\]
Similmente definiamo l'\textbf{$\alpha$-limite} considerando i limiti di successioni di tempi che vanno a $-\infty$.
\end{definition}
\begin{remark}
In realt\`a non \`e necessario chiedere successioni monotone. Per successioni generali che tendono a $\pm\infty$ basta passare a sottosuccessioni.
\end{remark}

\begin{example}
Seguono esempi di $\alpha$ e $\omega$ limiti semplici:
\begin{itemize}
\item Se $x$ \`e un punto fisso $\alpha(x)=\omega(x)=\cpa x$
\item Se $x$ \`e periodico $\alpha(x)=\omega(x)=\Oc(x)$
\end{itemize}
\end{example}

\begin{definition}[Omo-/etero-clino]
Siano $x_1$ e $x_2$ punti fissi e supponiamo $\cpa{x_1}=\alpha(y),\ \cpa{x_2}=\omega(y)$.\\ 
Se $x_1=x_2=x$ allora $y$ \`e detto \textbf{omoclino} di $x$.\\
Se $x_1\neq x_2$ allora $y$ \`e detto \textbf{eteroclino}.
\end{definition}

\begin{proposition}[Orbita limitata implica invarianza di $\omega$-limite]\label{OrbitaPositivaLimitataImplicaCompattezzaEInvarianzaOmegaLimite}
Se $x\in\R^d$ e $\Oc^+(x)$ \`e limitata allora $\omega(x)$ \`e non vuoto, compatto e invariante.
\end{proposition}
\begin{proof}
Sia $\cpa{\tau_j}\to+\infty$ tale che $\tau_1>0$ e $\tau_{j+1}>\tau_j$. 
Verifichiamo in linea preliminare la seguente uguaglianza:
\[\omega(x)=\bigcap_{j\geq 1}\ol{\Oc^+(\phi_{\tau_j}(x))}.\]
\setlength{\leftmargini}{0cm}  
\begin{itemize}
\item[$\boxed{\subseteq}$] Segue dal fatto che chiuso implica chiuso per successioni: per ogni $t\geq 0$ fissato vale definitivamente in $k$
\[\begin{cases}
y=\lim_k\phi_{t_k}(x)\\
\phi_{t_k}(x)\in \Oc^+(\phi_t(x))
\end{cases}\implies y\in \ol{\Oc^+(\phi_t(x))}\]
\item[$\boxed{\supseteq}$] Sia $y\in \bigcap_{j\geq 1}\ol{\Oc^+(\phi_{\tau_j}(x))}$. Per definizione di intersezione si ha che per ogni $j\geq 1$ abbiamo $y\in\ol{\Oc^+(\phi_{\tau_j}(x))}$. In particolare, per una caratterizzazione di chiusura in spazi metrici, esiste $t_j>0$ tale che
\[d(\phi_{t_j+\tau_j}(x),y)=d(\phi_{t_j}(\phi_{\tau_j}(x)),y)\leq \frac1j.\]
Segue che $\displaystyle\lim_{j\to+\infty}d(\phi_{t_j+\tau_j}(x),y)=0$, cio\`e $y\in \omega(x)$.
\end{itemize}
Verificata l'uguaglianza sopra, segue immediatamente che $\omega(x)$ \`e non vuoto e compatto (intersezione di compatti inscatolati non vuoti). Per mostrare l'invarianza mostriamo che $y\in\omega(x)\implies \phi_t(y)\in\omega(x)$ per ogni $t\in\R$.\\
Per definizione di $\omega$-limite esiste una successione $t_k\to+\infty$ tale che $\phi_{t_k}(x)\to y$. La tesi segue dalla seguente catena di identit\`a:
\[\phi_t(y)=\phi_t\pa{\lim_{k\to+\infty}\phi_{t_k}(x)}\overset{\text{continuit\`a in $t$}}=\lim_{k\to+\infty}\phi_{t+t_k}(x)\in\omega(x).\]
\end{proof}


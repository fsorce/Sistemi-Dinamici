\chapter*{Introduzione}


\begin{definition}[Sistema dinamico]
Un \textbf{sistema dinamico} \`e una terna $(X,G,S)$, dove $X$ \`e un insieme, $G$ un (semi)gruppo e $S=G\acts X$ una azione.\\
L'insieme $X$ \`e detto \textbf{spazio delle fasi}.
\end{definition}
\begin{remark}
In realt\`a vorremmo che l'azione abbia una regolarit\`a compatibile con la struttura dell'insieme in esame.
\end{remark}


\section*{Sistemi continui}
\begin{definition}[Equazione differenziale]
Un \textbf{sistema di equazioni differenziali} (in forma esplicita) \`e un'equazione della forma
\[\dot x=F(x),\quad x\in\R^d, \ F\in C^1(\R^d).\]
Un \textbf{problema di Cauchy} associato all'equazione differenziale sopra \`e dato da un sistema del tipo
\[\begin{cases}
\dd t{}x(t)=F(x(t))\\
x(0)=x_0
\end{cases}\]
Per il teorema di Cauchy-Lipschitz \`e ben definito il \textbf{flusso} $\phi:\R^d\times \R\to \R^d$ associato all'equazione differenziale definito come la funzione che rispetta 
\[\begin{cases}
\dd t{}\phi_t(x_0)=F(\phi_t(x_0))\\
\phi_0(x_0)=x_0
\end{cases}.\]
\end{definition}

\begin{remark}
Osserviamo che $\phi_0:\R^d\to \R^d$ \`e la mappa identit\`a e che $\phi_{\cdot}(x)$ \`e un omomorfismo per ogni $x$, cio\`e
\[\phi_t(\phi_s(x))=\phi_{s+t}(x).\]
\end{remark}

\begin{definition}[Sistema dinamico a tempo continuo]
Una terna $(\R^d,\R,\phi_t(\cdot))$ \`e un \textbf{sistema dinamico (a tempo) continuo} se $\phi_t(\cdot)$ \`e il flusso di un sistema di ODE della forma $\dot x=F(x)$ con $F(x)$ campo di vettori, $F\in C^k$ con $k\geq1$.
\end{definition}


\begin{example}[Equazioni differenziali su $\R$]
Consideriamo l'equazione
\[\dot x=ax.\]
Il nostro approccio standard \`e imporre una condizione di Cauchy $x(0)=x_0$ e troviamo la soluzione
\[x(t)=x_0e^{at}.\]
Ponendo $G=\R,\ X=\R$ e $S=G\acts X$ dove $S(t,x_0)=\phi_t(x_0)$ possiamo interpretare l'equazione sopra come un sistema dinamico.

\noindent
Se consideriamo ora un'equazione del tipo
\[\dot x=a(x+x^9)\]
diventa difficile risolvere l'equazione. Possiamo fare studi qualitativi.

\noindent Negli studi qualitativi per $X=\R$ sono utili le seguenti considerazioni
\begin{itemize}
\item Rintracciare punti fissi
\item Studiare il segno della derivata
\item Soluzioni diverse non si incrociano per unicit\`a locale
\item Teorema di approssimazione lineare.
\end{itemize}
\end{example}

\noindent Di solito ci interesseranno equazioni in $\R^2$ o $\R^3$. Seguono alcuni modi per trasformare alcuni tipi di equazioni differenziali nella forma $\dot \ulx=\ul F(\ulx)$\footnote{Non metter\`o mai pi\`u le barre se posso.}.
\begin{example}[Ordine superiore]
Consideriamo l'equazione
\[\ddot x=-kx,\quad k>0.\]
Ponendo $y=\dot x$ troviamo
\[\begin{cases}
\dot x=y\\
\dot y=-kx
\end{cases}\coimplies \dot{\spa{\emat{x\\y}}}=\mat{y\\-kx}.\]
\end{example}

\begin{example}[Sistema non autonomo]
Consideriamo l'equazione
\[\dot x=f(x)+g(t)\]
Possiamo allargare lo spazio delle fasi come segue
\[\begin{cases}
\dot x=f(x)+g(t)\\
\dot t=1
\end{cases}.\]
\end{example}

%\noindent Per completezza mostriamo un sistema dinamico continuo che non deriva da una equazione differenziale:
%\begin{example}
%Ricordiamo la seguente
%\begin{definition}[Gruppo speciale lineare]
%Il gruppo speciale lineare a coefficienti in $R$ e dimensione $d$ \`e
%\[SL(d,R)=\cpa{M\in M(d,R)\mid \det M=1}.\]
%\end{definition}
%\noindent
%Consideriamo il seguente sistema dinamico: 
%\[X=SL(2,\R),\ G=\cpa{\mat{e^t & 0\\ 0 & e^{-t}}\mid t\in \R}\quad \text{e}\quad S:\funcDef{G\times X}{X}{(a,g)}{ga}.\]

%Consideriamo invece $X=\quot{SL(2,\R)}{SL(2,\Z)}$ [FINISCI NEL CASO]
%\end{example}

\section*{Sistemi discreti}
Definiamo ora sistemi dinamici discreti
\begin{definition}[Sistema dinamico a tempo discreto]
Un \textbf{sistema dinamico (a tempo) discreto} \`e una terna $(X,\N,T)$ dove $X\in\{\R,\ [a,b],\ S^1\}$ e $T:X\to X$ continua. Il sistema dinamico \`e dato da $(X,\N, S)$ dove $S(n,x)=T^n(x)$.
\end{definition}
\begin{remark}
Se $T$ \`e invertibile possiamo estendere un sistema dinamico discreto sostituendo $\N$ con $\Z$ e ponendo (per $n\in\N^+$) $S(-n,x)=(T\ii)^n(x)$.
\end{remark}

\begin{example}[Successione per ricorrenza]
Una successione per ricorrenza \`e un sistema dinamico discreto
\[\begin{cases}
x_n=T(x_{n-1})\\
x_0=a
\end{cases}\]
\end{example}

\begin{example}[Mappa di Bernoulli]
Consideriamo
\[S^1=\cpa{z\in\C\mid \abs z=1},\quad T:\funcDef{S^1}{S^1}{z}{z^2}\]
Osserviamo che l'unico punto fisso \`e $0$, ma reiterando da un punto qualsiasi non c'\`e modo che la successione vi tendi. Inoltre la distanza tra punti si raddoppia ad ogni iterata (a meno del modulo). Questo \`e uno dei primi esempi di sistemi caotici.
\end{example}
\chapter{Basi di Dinamica Topologica}

Consideriamo $T:X\to X$ con $X$ compatto e $T$ continua. Con la notazione $T^n$ intendiamo sempre la $n$-iterata di $T$.\\
Per semplicit\`a poniamo $\N^+=\N\nz$.

\section{Punti periodici e coniugio topologico}
\begin{definition}[Orbita]
Sia $x\in X$, l'\textbf{orbita} di $x$ \`e l'insieme
\[\Oc(x)=\cpa{T^n(x)}_{n\in\N}.\]
\end{definition}
\begin{definition}[Punto periodico]
Un punto $x\in X$ \`e \textbf{periodico} se esiste $m\in\N^+$ tale che $T^m(x)=x$.\\ 
Se $x$ \`e periodico chiamiamo \textbf{periodo minimo} il numero
\[p=\min\cpa{m\in \N^+ \mid T^m(x)=x}.\]
\end{definition}
\begin{remark}
Un \textbf{punto fisso} \`e un punto periodico di periodo minimo 1 (cio\`e $T(x)=x$).
\end{remark}
\begin{remark}
Se $x$ \`e periodico con periodo minimo $p$ allora $\Oc(x)=\cpa{x,T(x),\cdots,T^{p-1}(x)}$.
\end{remark}

\begin{definition}[Punti definitivamente periodici]
Un punto $x\in X$ si dice \textbf{definitivamente periodico} se esiste $m\in\N^+$, tale che $T^{m}(x)$ \`e periodico ma $x$ non lo \`e.
\end{definition}

\begin{remark}
Il numero di punti fissi di periodo minimo $p$ \`e un multiplo di $p$, precisamente 
\[p\cdot \#\cpa{\text{Orbite disgiunte}}.\]
\end{remark}

\begin{definition}[Coniugio topologico]
Due insiemi $X_1,X_2$ tali che esistano $T_1:X_1\to X_1$ e $T_2:X_2\to X_2$ continue si dicono \textbf{coniugati topologicamente} se esiste $\vp:X_1\to X_2$ omeomorfismo tale che $\vp\circ T_1=T_2\circ\vp$, cio\`e commuta il diagramma
\[\begin{tikzcd}
	{X_1} & {X_1} \\
	{X_2} & {X_2}
	\arrow["{T_2}", from=2-1, to=2-2]
	\arrow["{T_1}", from=1-1, to=1-2]
	\arrow["\vp"', from=1-1, to=2-1]
	\arrow["\vp"', from=1-2, to=2-2]
\end{tikzcd}\]
\end{definition}
\begin{remark}
Con la notazione sopra, segue che $\vp\circ T_1^n=T_2^n\circ \vp$, quindi due sistemi coniugati topologicamente hanno la stessa dinamica.
\end{remark}

\begin{definition}[Insieme invariante]
Sia $(X,T,\N)$ un sistema discreto. $A\subseteq X$ \`e \textbf{positivamente invariante} se $T(A)\subseteq A$ e \textbf{invariante} se $T(A)=A$.
\end{definition}

\begin{remark}
A volte (fuori da questo corso) pu\`o essere utile definire l'invarianza come
\[T\ii(A)=A.\]
Questa nozione NON \`e equivalente alla precedente.
\end{remark}

\begin{definition}[$\omega$-limite]
Definiamo l'\textbf{$\omega$-limite} di $x$ come
\[\omega(x)=\cpa{y\in X\mid \exists n_k\nearrow+\infty\ t.c.\ \lim_{k\to+\infty}T^{n_k}(x)=y}.\]
Similmente definiamo l'\textbf{$\alpha$-limite}.
\end{definition}
Supponiamo da ora in poi che $X$ sia uno spazio metrico compatto.
\begin{proposition}[Orbita limitata restituisce $\omega$ compatto invariante]\label{OrbitaLimitataDaOmegaLimiteCompattoInvariante}
Se $\Oc(x)$ \`e limitata\footnote{servirebbe totalmente limitata perch\'e dobbiamo estrarre una sottosuccessioni di Cauchy nella dimostrazione. Se $X$ \`e una variet\`a reale di dimensione finita questa condizione \`e automaticamente verificata.} allora $\omega(x)$ \`e un compatto non vuoto. Se $T$ \`e continua allora $\omega(x)$ \`e invariante.
\end{proposition}
\begin{proof}
Come nel teorema (\ref{OrbitaPositivaLimitataImplicaCompattezzaEInvarianzaOmegaLimite}) ricaviamo che $\omega(x)=\bigcap_{n\in\N} \ol{\Oc(T^{n}(x))}$ da cui segue compattezza e $\omega(x)\neq \emptyset$.\\ 
Mostriamo l'invarianza:
\setlength{\leftmargini}{0cm}
\begin{itemize}
\item[$\boxed{T(\omega(x))\subseteq\omega(x)}$] Se $y\in \omega(x)$ allora troviamo una successione di istanti tale che $T^{n_k}(x)\to y$, dunque
\[T(y)=T(\lim T^{n_k}(x))\pasgnl={$T$ cont.}\lim T^{n_k+1}(x)\in \omega(x).\]
\item[$\boxed{T(\omega(x))\supseteq\omega(x)}$] Se $y\in \omega(x)$ allora troviamo una successione crescente di istanti tale che $T^{n_k}(x)\to y$. Poich\'e $\cpa{T^{n_k-1}(x)}\subseteq \Oc(x)$ esiste una sottosuccessione $k_j$ tale che $T^{n_{k_j-1}}(x)\to z\in X$\footnote{$\Oc(x)$ \`e (totalmente) limitata quindi possiamo estrarre una sottosuccessione di Cauchy, ma poich\'e $X$ \`e compatto questa successione converge a un punto di $X$.}. Per definizione di $\omega$-limite $z\in \omega(x)$ e 
\[T(z)=T(\lim T^{n_{k_j}-1}(x))=\lim T^{n_{k_j}}(x)=y\]
dove l'ultimo passaggio segue dal fatto che $T^{n_{k_j}}(x)$ \`e una sottosuccessione di $T^{n_k}(x)$.
\end{itemize}
\setlength{\leftmargini}{0.5cm}
\end{proof}


\section{Dinamica simbolica}
Sia $\Ac$ un alfabeto finito e sia $\Omega=\Ac^\N$. Imponiamo la topologia discreta su $\Ac$ e la topologia prodotto su $\Omega$. Per il teorema di Tychonoff segue dalla compattezza di $\Ac$ che $\Omega$ \`e compatto.\\
Lo spazio $\Omega$ \`e anche uno spazio metrico con distanza definita come segue:
\[d(\omega,\wt \omega)=\sum_{i=0}^\infty 2^{-i-1}\delta(\omega_i,\wt \omega_i),\]
dove $\delta(a,b)=\begin{cases}
1 & a\neq b\\
0 & a=b
\end{cases}$.

\begin{definition}[Shift]
Definiamo la mappa di \textbf{shift} come
\[\sigma:\funcDef{\Omega}{\Omega}{(\omega_i)_{i\in\N}}{(\omega_{i+1})_{i\in\N}}.\]
\end{definition}
\begin{remark}
La mappa di shift \`e continua perch\'e uniformemente continua.
\end{remark}

\section{Stabilit\`a locale dei punti fissi}
\begin{definition}[Punto attrattivo/repulsivo]
Un punto fisso $x\in X$ si dice \textbf{attrattivo} se esiste un intorno $U$ di $x$ tale che per ogni $y\in U$ si ha che $\cpa{T^n(y)}_{n\in\N}\subseteq U$ e $\displaystyle\lim_{n\to+\infty}T^n(y)=x$.\\
L'intorno $U$ come sopra \`e un \textbf{bacino di attrazione}.
\vspace{0.25cm}

\noindent
Un punto fisso $x\in X$ si dice \textbf{repulsivo} se esiste un intorno $U$ di $x$ tale che per ogni $y\in U\bs \cpa x$ esiste $n(y)\in\N$ tale che $T^{n(y)}(y)\notin U$.
\end{definition}

\begin{definition}[Orbita attrattiva/repulsiva]
Sia $x$ punto periodico di periodo minimo $p$, allora $\Oc(x)=\cpa{x,\cdots, T^{p-1}(x)}$ \`e \textbf{attrattiva} (rispettivamente \textbf{repulsiva}) se $x$ \`e attrattivo (rispettivamente repulsivo) per $T^p$.
\end{definition}

\begin{definition}[Punto iperbolico]
Supponiamo che $X\in\cpa{[0,1], S^1, \R, (a,b), [a,b), (a,b]}$ e $T\in C^1(X,X)$. Un punto fisso $x\in X$ si dice \textbf{iperbolico} se $\abs{T'(x)}\neq 1$\footnote{L'idea \`e che $T(y)-x=T(y)-T(x)=T'(x)(y-x)+o(y-x)$, dunque se $\abs{T'(x)}\neq 1$ allora il termine lineare ci dice se per $y$ abbastanza vicino a $x$ vale $\abs{T(y)-x}>\abs{y-x}$ o viceversa.}.
\end{definition}

\begin{proposition}[Relazione tra punti iperbolici e attrattivit\`a]\label{RelazioneTraPuntiIperboliciEAttrattivita}
Sia $T\in C^1(X,X)$ e $x$ un punto fisso iperbolico. Allora 
\begin{align*}
\abs{T'(x)}<1&\implies x\text{ \`e attrattivo}\\
\abs{T'(x)}>1&\implies x\text{ \`e repulsivo.}
\end{align*}
\end{proposition}
\begin{proof}
Studiamo i due casi
\setlength{\leftmargini}{0cm}
\begin{itemize}
\item[$\boxed{\abs{T'(x)}<1}$] Sia $c\in (\abs{T'(x)},1)$ e consideriamo $\delta>0$ tale che $\abs{T'(y)}\leq c$ per ogni $y\in [x-\delta,x+\delta]$. Sia ora $y\in (x-\delta,x+\delta)$, segue che esiste $\xi\in (x,y)$ tale che
\[\abs{T(y)-x}=\abs{T'(\xi)}\abs{y-x}\leq c\abs{y-x}.\]
Segue in particolare che $T(y)\in (x-\delta,x+\delta)$. Ripetendo questo procedimento si ha che $\abs{T^2(y)-x}\leq c\abs{T(y)-x}\leq c^2\abs{y-x}$. Procedendo per induzione
\[\abs{T^n(y)-x}\leq c^n\abs{y-x}\underset{n\to+\infty}{\to} 0,\]
cio\`e $x$ \`e attrattivo con bacino di attrazione $(x-\delta,x+\delta)$.
\item[$\boxed{\abs{T'(x)}<1}$] Sia $c\in (1,\abs{T'(x)})$ e sia $\delta>0$ tale che $\abs{T'(y)}\geq c$ per ogni $y\in [x-\delta,x+\delta]$. Consideriamo ora $y\in (x-\delta,x+\delta)$. Supponiamo per assurdo che $T^n(y)\in (x-\delta,x+\delta)$ per ogni $n$. Con un ragionamento analogo a prima ricaviamo
\[\delta>\abs{T^n(y)-x}\geq c^n\abs{y-x}\underset{n\to+\infty}{\to}+\infty,\]
che \`e assurdo.
\end{itemize}
\setlength{\leftmargini}{0.5cm}
\end{proof}

\begin{remark}[Derivata delle iterate]
Osserviamo che
\begin{align*}
(T^p)'(x)=&\dd x{}\pa{T^{p-1}(T(x))}=(T^{p-1})'(T(x))T'(x)=\\
=&((T^{p-2})'(T(T(x)))T'(T(x)))T'(x)=\cdots\\
\cdots=&\under{=1}{(T^{p-p})'(T^p(x))}\prod_{k=0}^{p-1}T'(T^k(x))=\\
=&\prod_{k=0}^{p-1}T'(T^k(x)).
\end{align*}
In particolare se $x$ \`e un punto di periodo $p$ segue riordinando i fattori della formula sopra che $(T^p)'(T^k(x))=(T^p)'(x)$.
\end{remark}
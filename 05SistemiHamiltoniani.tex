\chapter{Integrali Primi e Sistemi Hamiltoniani}
In questo capitolo introduciamo una classe di sistemi che ammettono integrale primo (e quindi ci permettono di fare uno studio globale guardando le curve di livello di questo). A volte, anche se un sistema non \`e della forma trattata in questo capitolo, \`e possibile studiarne il comportamento approssimandolo prima ad uno di questi e poi perturbando la soluzione trovata in modo opportuno.
\section{Sistemi Hamiltoniani}
\begin{notation}
Quando consideriamo lo spazio $\R^{2d}$ in questo capitolo indichiamo le coordinate come $(x,y)=(q,p)\in\R^{2d}$ con $x=q\in\R^d$ e $y=p\in\R^d$. In genere $q$ \`e detto \textbf{coordinata generalizzata} e $p$ \`e detto \textbf{momento generalizzato}.\\
Con la notazione $\pp xH$ intendiamo il vettore la cui componente $i$-esima \`e $\ppxi iH$. Similmente per $\pp yH$.
\end{notation}
\begin{definition}[Sistema Hamiltoniano]
Un sistema di ODE su $\R^{2d}$ \`e un \textbf{sistema Hamiltoniano} se \`e della forma
\[\mat{-\dot y\\\dot x}=\nabla H\coimplies \begin{cases}
\displaystyle \dot x_i=\pp{y_i}H\\\\
\displaystyle \dot y_i=-\pp{x_i}H
\end{cases}\text{per ogni }i\in\cpa{1,\cdots,d},\]
dove $H:\R^{2d}\to\R$ \`e una funzione di classe $C^2(\R^{2d})$ detta \textbf{funzione Hamiltoniana} del sistema.\\
Il numero $d$ \`e detto numero di \textbf{gradi di libert\`a}.
\end{definition}
\begin{remark}
Ad ogni funzione $H:\R^{2d}\to\R$ di classe $C^2$ \`e associato un sistema Hamiltoniano.
\end{remark}

\begin{proposition}[Hamiltoniana \`e un integrale primo]\label{HamiltonianaEIntegralePrimo}
Se $H$ \`e una funzione Hamiltoniana allora \`e un integrale primo del sistema ad essa associato.
\end{proposition}
\begin{proof}
Basta calcolare $\dot H$.
\begin{align*}
\dot H(x,y)=&\nabla H(x,y)\cdot \mat{\displaystyle \pp {y}H \\\\ \displaystyle -\pp {x}H}=\\
=&\sum_{i=1}^d\pa{\pp{x_i}H\pp{y_i}H+\pp{y_i}H\pa{-\pp{x_i}H}}=0.
\end{align*}
\end{proof}

\begin{example}
Un sistema con due gradi di libert\`a \`e dato dal doppio pendolo, il quale \`e interamente determinato da due angoli. Lo spazio delle fasi \`e $T^2\times \R^2$ dove $T^2=S^1\times S^1$ parametrizza i due angoli e $\R^2$ le velocit\`a angolari.\\
Osserviamo intuitivamente che $\dim \cpa{H=c}=3$, che vedremo essere la minima dimensione che permette orbite caotiche.
\end{example}
\begin{example}[Un grado di libert\`a e mezzo]
Se un sistema ha un grado di libert\`a ma non \`e autonomo possiamo reinterpretarlo come un sistema con due gradi di libert\`a dove abbiamo aggiunto il tempo e questi sistemi possono ancora presentare comportamenti caotici. Un esempio sono oscillatori armonici perturbati periodicamente.
\end{example}

\begin{theorem}[Liouville]\label{TeoremaLiouville}
Sia $H:\R^{2d}\to \R$ di classe $C^3$ e sia $\phi_t$ il flusso del sistema Hamiltoniano associato. Allora per ogni $A\subseteq \R^{2d}$ misurabile si ha che per ogni $t\in\R$
\[\Lc(A)=\Lc(\phi_t(A)).\]
\end{theorem}
\begin{proof}
Osserviamo che
\[\dd t{}\phi_t(x,y)=F(\phi_t(x,y))=\mat{\displaystyle\pp yH(\phi_t(x,y))\\\\\displaystyle-\pp xH(\phi_t(x,y))}.\]
Poich\'e $H$ \`e di classe $C^3$, si ha che $\phi:(x,y,t)\mapsto\phi_t(x,y)$ \`e di classe $C^2$ \footnote{Poich\'e $H$ \`e $C^3$ ne deriva che $F$ \`e $C^2$. Dato che $\phi'=F(\phi)$ si ha per il teorema del differenziale totale che $\phi$ \`e almeno $C^2$ in ogni entrata.}, quindi
\[\dd t{}\pa{\Dc(\phi_t(x,y))}\pasgnl={regolarit\`a}\Dc \under{=F(\phi_t(x,y))}{\pa{\dd t{}\phi_t(x,y)}}=\Dc F(\phi_t(x,y))\Dc (\phi_t(x,y)).\]
Poich\'e $\Dc\phi_0(x,y)=I$, possiamo integrare l'identit\`a precedente\footnote{\`e come se stessimo risolvendo insieme tutti i sistemi della forma \[\dd t{}{}v(t)=\Dc F(\phi_t(x,y))v(t),\quad\text{per } v(t)=\Dc(\phi_t(x,y))e_i:\R\to \R^{2d}.\]} per trovare
\[\Dc(\phi_t(x,y))=\exp\pa{\int_0^t\Dc F(\phi_s(x,y))ds}.\]
Segue da una identit\`a nota\footnote{$\det(e^A)=e^{\tr A}$} che
\begin{align*}
\det{\Dc(\phi_t(x,y))}=&\exp\pa{\int_0^t\tr\pa{\Dc F(\phi_s(x,y))}ds}.
\end{align*}
Per il teorema di Schwarz 
\[\pp{x_i\del y_i}{^2H}=\pp{y_i\del x_i}{^2H},\] 
dunque $\tr(\Dc F)=0$, da cui
\[\tr\pa{\Dc F(\phi_s(x,y))}=0.\]
Segue che
\[\det{\Dc(\phi_t(x,y))}=e^0=1.\]
Concludiamo con il seguente conto
\begin{align*}
\Lc(\phi_t(A))=&\int_{\phi_t(A)}1d\Lc=\int_A\abs{\det \Dc \phi_t(x,y)(x,y)}d\Lc=\\
=&\int_A 1d\Lc=\Lc(A).
\end{align*}
\end{proof}

\subsection{Sistemi meccanici con un grado di libert\`a}
\begin{definition}[Sistema meccanico]
Un \textbf{sistema meccanico} \`e un sistema Hamiltoniano della forma
\[H(x,y)=\frac12y^\top A y+V(x)\]
con $V:\R^d\to\R$ e $A\in Sym(d,\R)$. La funzione $V$ \`e detta \textbf{potenziale}.
\end{definition}

\noindent Studiamo un sistema meccanico con un grado di libert\`a:\\
A meno di riscalamento poniamo $A=\mat1$.
\[\begin{cases}
\displaystyle\dot x=\pp yH=y\\\\
\displaystyle\dot y=-\pp xH=-V'(x)
\end{cases}\]
Osserviamo che i punti fissi sono $\cpa{(x,0)\mid V'(x)=0}$. La matrice Jacobiana \`e data da
\[\Dc F(x,y)=\mat{0 & 1\\-V''(x) &0}\implies \tr \Dc F=0,\ \det \Dc F=V''(x).\]
Se $V''(x)=0$ siamo in un caso degenere e il linearizzato non ci aiuta.\\
Se $V''(x)<0$ il punto $(x,0)$ \`e una sella.\\
Se $V''(x)>0$ il punto $(x,0)$ \`e un punto non iperbolico di tipo centro.
\vspace*{0.5cm}

\noindent Il teorema di linearizzazione ci informa solo sul caso $(x,0)$ sella. Studiamo la stabilit\`a del caso di tipo centro.



\begin{proposition}[Caratterizzazione dei punti fissi in sistema meccanico ad un grado di libert\`a]\label{CaratterizzazionePuntiFissiSistemaMeccanico1GradoLiberta}
I punti fissi non degeneri\footnote{$(x_0,0)$ con $V'(x_0)=0$ e $V''(x_0)\neq 0$} di un sistema meccanico ad un grado di libert\`a sono selle o centri\footnote{non solo nel linearizzato}.
\end{proposition}
\begin{proof}[Idea di dimostrazione.]
Se $(x_0,0)$ \`e una sella nel linearizzato allora la tesi segue per Hartmann-Grobman (\ref{TeoremaHartmanGrobman}). Supponiamo dunque che $(x_0,0)$ sia un centro lineare tale che $V''(x_0)\neq 0$.\\
Se $(x_0,0)$ fosse instabile, esisterebbe un $\e>0$ tale che per ogni $\delta>0$ esiste $z_0\in B_\delta((x_0,0))$ tale che $\norm{\phi_t(z_0)-(x_0,0)}>\e$. Poniamo
\[m_\e=\min_{\del{B_\e((x_0,0))}} H\]
e consideriamo un $\e$ abbastanza piccolo in modo tale che $m_\e\sup_{B_\e((x_0,0))} H$, che sappiamo esistere perch\'e da noti criteri sulle derivate prime e seconde $(x_0,0)$ \`e un minimo locale di $H$.
Poich\'e le orbite del sistema corrispondono alle curve di livello di $H$ abbiamo trovato una contraddizione in quanto per $\e$ abbastanza piccolo le orbite non possono attraversare $\del B_\e((x_0,0))$.\\
Se $(x_0,0)$ fosse asintoticamente stabile esisterebbe un intorno di $(x_0,0)$ che converge a $(x_0,0)$, in particolare l'area di questo intorno non si conserva, contraddicendo il teorema di Liouville (\ref{TeoremaLiouville}).
\end{proof}
\medskip

\noindent
Nel caso di $V'(x_0)=V''(x_0)=0$ spesso ci aspettiamo orbite omocline dove il punto fisso \`e una cuspide. Pi\`u precisamente\footnote{Questa motivazione non \`e stata data a lezione.}, applicando il teorema di Taylor:
\[\dd xy=-\frac{V'(x)}{y}=-\frac{V'(x)}{\pm \sqrt{2(V(x_0)-V(x))}}=-\frac{o(x-x_0)}{\pm\sqrt{2o((x-x_0)^2)}}=o(1)\underset{x\to x_0}{\longrightarrow}0,\]
cio\`e troviamo una cuspide.
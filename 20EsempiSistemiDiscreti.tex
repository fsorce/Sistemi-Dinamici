\chapter{Esempi di sistemi discreti}
\section{Motivazione: le mappe di Poincar\'e}
Proviamo a capire quando un sistema di equazioni differenziali porta un'orbita a tornare vicino a se stessa.

\begin{definition}[Mappa di Poincar\'e]
Sia $M$ una variet\`a e sia $\Sigma$ una sua sottovariet\`a di codimensione 1 tale che esiste $U\subseteq \Sigma$ con la seguente propriet\`a:\\
se $P\in U$ allora esiste $\ol t>0$ tale che $t\in (0,\ol t)\implies \phi_t(P)\notin U$ e $\phi_{\ol t}(P)\in U$\footnote{$\ol t$ \`e l'istante del ``primo ritorno"}.\\
Definiamo la \textbf{mappa di Poincar\'e} come
\[P_U:\funcDef{U}{U}{(x,y)}{\phi_{\ol t}(x,y)}.\]
\end{definition}
\begin{remark}
Ponendo $P_\Sigma^0=id$ e
\[P_\Sigma^n=\under{n\text{ volte}}{P_\Sigma\circ\cdots\circ P_\Sigma}\]
stiamo definendo un sistema discreto $(\Sigma,P_\Sigma,\Z)$\footnote{$P_\Sigma^{-1}$ \`e definita perch\'e sono partito da un flusso e posso prendere tempi negativi in un flusso}.
\end{remark}

\begin{example}[Moto sul toro piatto]
Sia $T^2=\R^2/\Z^2$. Un moto geodetico sul toro (piatto) si pu\`o pensare come
\[t\mapsto \phi_t(x,y)=(x,y)+t(v_x,v_y)\mod{\Z^2}.\]
Sia $\Sigma=U=S^1=\quot{[0,1]\times\cpa0}{(0,0)\sim(1,0)}$. Per la geometria del toro questa sottovariet\`a di $T^2$ ha le propriet\`a richieste per definire la mappa di Poincar\'e (addirittura sappiamo che $\ol t=1/v_y$).\\
Esplicitamente troviamo che 
\[P_\Sigma:\funcDef{S^1}{S^1}{t}{t+\al \mod 1}\]
dove $\al=v_x/v_y$. Osserviamo che a meno di traslare modulo 1, tutte le orbite sono determinate dall'orbita di $0$.
\setlength{\leftmargini}{0cm}
\begin{itemize}
\item[$\boxed{\al\in\Q}$] Se $\al=\frac pq$ ridotta ai minimi termini allora $P_\Sigma^q(0)=0+p=0 \mod 1$ e le orbite sono periodiche.
\item[$\boxed{\al\in\R\bs\Q}$] Evidentemente non troviamo un'orbita periodica. \`E possibile mostrare che in realt\`a l'orbita \`e densa.
\end{itemize}
\setlength{\leftmargini}{0.5cm}
\end{example}


\begin{definition}[Semipiano di Poincar\'e]
Consideriamo la regione $[-1,1]\times [0,+\infty]\bs D^1$ e identifichiamo i lati come in figura
\begin{figure}[!htb]
    \centering
    \includegraphics[width=9cm]{Immagini/ModularGroup-FundamentalDomain-01.png}
    \caption{Dominio fondamentale e qualche geodetica.}
    \label{SemipianoPoincare}
\end{figure}

\noindent
Le geodetiche sono le intersezioni dell'oggetto con rette verticali o semicirconferenze perpendicolari all'asse $x$.
Chiamiamo $\Mc$ questo spazio.
\end{definition}
\begin{example}[Geodetiche sul piano di Poincar\'e]
A parte le geodetiche corrispondenti a rette verticali, ogni geodetica che incontra la regione lo fa passando per l'asse $y$ (e quindi lo fa ad un certo angolo). Possiamo associare ad ogni coppia punto di $\Mc$ e angolo una geodetica di $\Mc$ (quella passante per il punto che incontra la verticale a quell'angolo)
\[\Mc\times S^1\ni ((x,y),\theta)\mapsto g_t((x,y),\theta).\]
Sia $\Sigma=\cpa{x=0,y>1}\times S^1$. \`E possibile definire $P_\Sigma$ e si da il caso che questa mappa di Poincar\'e \`e pi\`u facile da studiare rispetto al sistema originale.
\end{example}

\section{Endomorfismi del cerchio e mappa di Bernoulli}
\begin{definition}[Endomorfismi lineari del cerchio]
Sia $m\in\R$, gli endomorfismi lineari del cerchio sono quelli della forma\footnote{Il caso $m=2$ restituisce la \textbf{mappa di Bernoulli}.}
\[T_m:\funcDef{S^1}{S^1}{x}{mx\mod 1}.\]
\end{definition}

\begin{example}[Punti fissi della mappa di Bernoulli]
Cerchiamo graficamente punti fissi e periodici di $T_2=T$
\begin{figure}[!htb]
    \centering
    \includegraphics[width=5cm]{Immagini/Bernoulli.png}
    \caption{Un tipo di grafico utile la mappa di Bernoulli. In grigio troviamo l'identit\`a, in rosso $T_2$ e in blu $T_2^2$.}
\end{figure}

\noindent Evidentemente l'unico punto fisso \`e $0$ (dal disegno sarebbero $0$ e $1$, ma $0\equiv 1\mod 1$).\\
Cerchiamo ora punti con periodo minimo $2$, cio\`e
\[T^2(x)=x,\quad T(x)\neq x.\]
Ricordiamo che
\[T(x)=\begin{cases}
2x & 0\leq x< \frac12\\
2x-1 & \frac12\leq x<1
\end{cases}\implies
T^2=\begin{cases}
4x & 0\leq x<\frac14\\
2(2x)-1=4x-1 & \frac14\leq x<\frac12\\
2(2x-1)=4x-2 & \frac12\leq x<\frac34\\
2(2x-1)-1=4x-3 & \frac34\leq x<1
\end{cases}\]
Graficamente vediamo che ci sono due punti di periodo $2$, e questi sono $\frac13$ e $\frac23$. Osserviamo che i numeri della forma $3\ii\cdot 2^{-k}$ sono definitivamente periodici.\\
Cerchiamo i punti di periodo minimo $3$. Graficamente notiamo che sono $6$
    
\begin{figure}[!htb]
    \centering
    \includegraphics[width=5cm]{Immagini/Bernoulli_iterata.png}
    \caption{Rappresentazione grafica di $T^3_2$.}
\end{figure}

\noindent
In generale per come funzionano le iterate di $T_2$ si ha che il numero di punti di periodo (eventualmente non minimo) $n$ \`e $2^n-1$.
\end{example}

\begin{example}[Perdo il controllo]
Consideriamo la mappa
\[T:\funcDef{S^1}{S^1}{x}{10x \mod 1}.\]
Poniamo $\Ac=\cpa{0,\cdots,9}$ e $\Ac_n=\pa{\frac n{10},\frac{n+1}{10}}$ per $n\in\Ac$.\\
Sia $x\in \R\bs \Q$ e diamo la seguente mappa
\[\vp:x\mapsto(\omega_0,\omega_1,\cdots)\in \Ac^{\N}\]
dove $\omega_n=k\coimplies T^n(x)\in \Ac_k\coimplies \lfloor10 T^n(x)\rfloor=k$ e per ricorsione vediamo che $\omega_n$ \`e la $n$-esima cifra decimale di $x$, cio\`e
\[x=0.\omega_0\omega_1\omega_2\cdots.\]
Per rispondere ad una domanda del tipo ``$T^{1000}(x)\in \Ac_i?$" devo sapere la 1000-esima cifra di $x$. Se ora considero $x+\e$ al posto di $x$, le informazioni che avevamo su $x$ non dicono pi\`u nulla sul comportamento di $x+\e$ oltre un certo passo\footnote{se $i>-\log_{10} \e$ la risposta a ``$T^{1000}(x)\in \Ac_i?$" e quella a ``$T^{1000}(x+\e)\in \Ac_i?$" sono indipendenti.}.
\end{example}
    
\begin{remark}[Piccola parentesi statistica]
Sia $x=0.x_1x_2\cdots$ potremmo chiederci, fissata una cifra $k$ se
\[Prob\cpa{\lim_{N\to+\infty}\frac{\#\cpa{i\in \cpa{0,\cdots, N-1}\mid x_i=k}}{N}=\frac1{10}}=1\]
ed effettivamente \`e vero. Segue dunque che
\[Prob\cpa{\lim_{N\to+\infty}\frac{\#\cpa{i\in \cpa{0,\cdots, N-1}\mid x_i=k}}{N}=\frac1{9}}=0\]
anche se non \`e un insieme vuoto\footnote{per esempio posso fissare ogni nona cifra a $k$ e completare le altre con cifre a caso diverse da $k$.}.
\end{remark}

\begin{example}[Espansione binaria tramite la mappa di Bernoulli]
Sia $T_2:S^1\to S^1$, $T_2(x)=2x\mod 1$.\\
Sia $I_0=[0,\frac12)$, $I_1=[\frac12,1)$ e $\Omega=\cpa{0,1}^\N$. Consideriamo la mappa $\vp:S^1\to \Omega$ data da:
\[x\mapsto (\omega_0(x),\omega_1(x),\cdots),\quad \omega_i(x)=\begin{cases}
0 & T^i(x)\leq \frac12\\
1 & T^i(x)\geq\frac12
\end{cases}\]
Osserviamo che $\sigma\circ \vp=\vp\circ T_2$ ma $\vp$ non \`e un omeomorfismo (o ben definita), infatti
\[\vp\ii(1,0,0,0,\cdots)=\cpa{\frac12}=\vp\ii(0,1,1,1,1,\cdots).\]
Quello che sta succedendo che abbiamo trovato due serie della seguente forma che convergono allo stesso valore
\[x=\sum_{i\geq 0}\frac{\omega_i(x)}{2^{i+1}}.\]
    
\end{example}


\section{Mappa Logistica e Tenda}    
\begin{definition}[Mappa logistica]
Una funzione continua $T_\la:[0,1]\to[0,1]$ si dice \textbf{logistica} se \`e della forma\footnote{Le restrizioni $0\leq \la\leq 4$ servono per garantire che effettivamente $[0,1]$ sia il codominio.}
\[T_\la(x)=\la x(1-x),\quad 0\leq \la\leq 4.\]
\end{definition}

\begin{figure}[!htb]
    \centering
    \includegraphics[width=6cm]{Immagini/Logistica.png}
    \caption{Rappresentazione di due mappe logistiche.}
\end{figure}


\begin{example}[Mappa logistica]
Osserviamo che $T_\la$ ha un massimo in $\frac12$.\\
Cerchiamo i punti fissi:
\[\la x(1-x)=x\coimplies x((\la-1)-\la x)=0,\]
quindi i punti fissi sono $0$ e $1-\la\ii$, dove il secondo punto fisso si presenta solo se compreso tra $0$ e $1$, cio\`e solo se $\la\geq 1$.\\
Calcoliamo quando questi punti fissi sono iperbolici:\\
\[T_\la'(x)=\la-2\la x.\]
Per $\la<1$ si ha che $0$ \`e iperbolico attrattivo e che per $\la>1$ \`e iperbolico repulsivo.\\
Se $\la=1$ allora i due punti fissi coincidono e sono non iperbolici.
\[T'_\la(1-\la\ii)=2-\la\]
quindi $1-\la\ii$ \`e iperbolico per $\la\neq 1$ o $3$.\\
Studiamo ora i casi di $\la=1$ e $\la=3$:
\setlength{\leftmargini}{0cm}
\begin{itemize}
\item[$\boxed{\la=1}$] Calcoliamo che $T_\la''(x)=-2\la<1$, quindi, poich\'e $T_\la'(0)=\la=1$ per il criterio per punti non iperbolici (\ref{CriterioPuntiNonIperboliciDerivataPositiva}) si ha che $0$ \`e attrattivo.
\item[$\boxed{\la=3}$] Il questo caso $1-\la\ii=2/3$, inotlre $T'(x)=3-6x$, $T''(x)=-6$ e $T'''(x)=0$. Segue che $ST(x)=0-\frac32(6/(3-6x))^2<0$, in particolare $ST(2/3)<0$ e quindi \`e un punto fisso attrattivo per il criterio per punti non iperbolici con derivata $-1$ (\ref{CriterioPuntiNonIperboliciDerivataNegativa}).
\end{itemize}
\setlength{\leftmargini}{0.5cm}
\end{example}





\begin{definition}[Mappa tenda]
Una funzione continua $T_a:[0,1]\to[0,1]$ si dice \textbf{tenda} se \`e della forma
\[T_a(x)=\begin{cases}
ax & 0\leq x< \frac12\\
a(1-x) & \frac12\leq x\leq 1
\end{cases},\quad 0\leq a\leq 2.\]
\end{definition}

\begin{example}[Insiemi invarianti per mappa tenda e mappa lineare]
Sia $T_a$ la mappa tenda come sopra e $T_b:[0,1]\to [0,1]$ la mappa lineare $T_b(x)=bx$ (poniamo $b\in [0,1]$).
\begin{figure}[!htb]
    \centering
    \includegraphics[width=5cm]{Immagini/Insieme_invariante_tenda.png}
    \caption{Grafico di una mappa tenda}
\end{figure}
Graficamente osserviamo che $T_a([0,1])=[0,\frac a2]$, quindi $[0,\frac a2]$ \`e positivamente invariante, ma \`e invariante solo se $a\geq 1$.
\begin{figure}[!htb]
    \centering
    \includegraphics[width=5cm]{Immagini/Insieme_invariante_lineare.png}
    \caption{Grafico di una mappa lineare}
\end{figure}
Osserviamo che $T_b([0,1])=[0,b]$, quindi $[0,b]$ \`e positivamente invariante, ma \`e invariante solo per $b=1$.
\end{example}

\begin{example}[Dinamica della mappa tenda]
Sia $T=T_a$ la mappa tenda di parametro $a$. 
\begin{figure}[!htb]
    \centering
    \includegraphics[width=5cm]{Immagini/Tipi_mappa_tenda.png}
    \caption{Grafico della mappa tenda per tre valori di $a$.}
\end{figure}

\noindent
Osserviamo graficamente che 
\[\text{Punti fissi}=\begin{cases}
\cpa{0} &a\in (0,1)\\
\spa{0,\frac12} &a=1\\
\cpa{0,\frac a{1+a}} &a>1
\end{cases}\]
Studiamo il comportamento di $0$ e $\frac a{1+a}$ (il secondo solo nel caso di $a\geq 1$)
\setlength{\leftmargini}{0cm}
\begin{itemize}
\item[$\boxed{0}$] Osserviamo che
\[\abs{T'(0)}=a,\]
quindi $0$ \`e iperbolico per $a\neq 1$. L'attrattivit\`a dipende dal segno di $a-1$.
\item[$\boxed{\frac a{1+a},\ a\geq 1}$] Anche in questo caso
\[\abs{T'\pa{\frac a{1+a}}}=a,\]
quindi questo punto \`e repulsivo per $a>1$ mentre per $a=1$ il punto non \`e iperbolico.
\end{itemize}
\setlength{\leftmargini}{0.5cm}
Studiamo la dinamica al variare di $a$
\setlength{\leftmargini}{0cm}
\begin{itemize}
\item[$\boxed{a=1}$] Per $x\in \spa{0,\frac12}$ si ha che $x$ \`e un punto fisso. Se $x\in (\frac12,1]$ allora $T(x)\in [0\frac12)$, quindi i punti sono ``definitivamente fissi".
\item[$\boxed{a<1}$] L'unico punto fisso \`e $0$ e $\omega(x)=\cpa{0}$ per ogni $x$.
\item[$\boxed{a>1}$] Abbiamo due punti fissi, ma sono entrambi repulsivi. Proviamo a studiare l'iterata seconda
\begin{figure}[!htb]
    \centering
    \includegraphics[width=5cm]{Immagini/mappa_tenda_iterata.png}
    \caption{Grafico della mappa tenda e della sua seconda iterata per $a>1$. In nero sono evidenziati i punti fissi di $T$ mentre con un cerchio sono evidenziati i punti di periodo minimo $2$ (visti come punti fissi di $T^2$).}
\end{figure}

Notiamo che se $x_1$ \`e uno dei punti fissi di $T$ allora $\abs{(T^2)'(x_1)}=a^2>1$, quindi continuano ad essere repulsivi anche per $T^2$ (non esistono dunque orbite attrattive di periodo 2).
\end{itemize}
\setlength{\leftmargini}{0.5cm}
\end{example}
    
\begin{remark}[Esempio di coniugio topologico]
La mappa logistica $T_4$ \`e coniugata topologicamente alla mappa tenda $T_2$.
\end{remark}
\begin{proof}
Basta considerare $\vp:[0,1]\to [0,1]$ data da
\[\vp(x)=\sin^2\pa{\frac\pi2x}\]
e notare che $\vp\circ T_2=T_4\circ \vp$\footnote{entrambe le composizioni assumono il valore $\sin^2(\pi x)$. Ricordiamo che $2\sin(\theta)\cos(\theta)=\sin(2\theta)$.}.
\end{proof}

\section{Sistemi caotici}
\begin{example}[Oscillatore armonico perturbato]
Sia $f:\R\to \R$ una funzione $1$-periodica. Immaginiamo una palla che rimbalza su un piatto la cui altezza varia come $f$. Per semplificarci la vita possiamo immaginare che l'urto avvenga sempre in $x=0$, tanto l'unica cosa che conta \`e $\dot f$ (l'impulso). Siano $t_0,t_1,\cdots,t_n$ i tempi di urto ($t_0=0$) e siano $v_0,\cdots,v_n$ le velocit\`a dopo l'urto. Conoscendo questi dati possiamo ricostruire tutta la dinamica.
\[T:\funcDef{[0,+\infty)\times (0,+\infty)}{[0,+\infty)\times (0,+\infty)}{(t_{n},v_n)}{(t_{n+1},v_{n+1})}.\]
Calcoliamo
\[\begin{cases}
t_{n+1}=t_n+h(v_n)\\
v_{n+1}=v_n+2\dot f(t_{n+1})
\end{cases},\quad h(v_n)=\frac 2gv_n\]
Osserviamo inoltre che
\[T(t_n+1,v_n)=(t_n+1+h(v_n), v_n+2\dot f(t_n+1 h(v_{n+1})))=T(t_n,v_n)+(1,0),\]
quindi in realt\`a possiamo considerare $S^1$ al posto di $[0,+\infty)$ per i tempi.\\
Questo sistema \`e difficile da trattare e presenta molti problemi aperti.
\end{example}
